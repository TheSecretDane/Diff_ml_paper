\section{Introduction}

Pricing of financial derivates is a fundamental problem in quantitative finance. A financial derivative is a contract whose value is derived from the performance of an underlying asset, index, or interest rate. Common examples include options, futures, and swaps. The accurate pricing of these instruments is crucial for risk management, investment strategies, and market efficiency. Barring the exception of simple cases like the Black-Scholes model, pricing financial derivatives often involves complex mathematical models and numerical methods. Monte Carlo simulation is one such numerical method that is widely used due to its flexibility and ability to handle high-dimensional problems. However, Monte Carlo methods can be computationally expensive, especially when high precision is required due to the estimators standard error being $1/\sqrt{N}$ convergent. By leveraging advancements in machine learning, we can potentially enhance the efficiency and accuracy of derivative pricing. This paper explores the integration of Monte Carlo simulation with differential machine learning techniques (MCDML) introduced in the CITEDIFFpaper to improve the pricing of financial derivatives. 